%------------------------------------%
%  file: cv.tex                      %
%  created at: 12 Nov 2023           %
%  compilation: `xelatex cv.tex`     %
%------------------------------------%
\documentclass{article}
\usepackage{titling}
\usepackage{hyperref}
\usepackage{titlesec}
\usepackage{multicol}
\usepackage{tasks}
\usepackage[margin={9em,5em,9em,5em}]{geometry}
\usepackage{xepersian}
%%
% fonts
\settextfont{Yas}
\setdigitfont{Yas}
\setlatintextfont{EB Garamond}
%%
% title format & spacing
\titleformat{\section}[frame]
{\Large}{}{.2em}{\filcenter\lowercase}[]
%
\titleformat{\subsection}
{\large\bfseries}{}{-.8em}{$\bullet$ }
%
\titleformat{\subsubsection}[runin]
{}{}{0em}{}[ ---]
%
\titlespacing{\subsection} {0em}{1em}{.3em}
\titlespacing{\subsubsection} {0em}{.25em}{1em}
%%
% line spacing
\renewcommand{\baselinestretch}{1.2}
% make email
\newcommand{\mkemail}[1]{\href{mailto: {#1}}{\texttt{#1}}}
% make telegram link
\newcommand{\mktelegram}[1]{\href{https://t.me/#1}{\texttt{@#1}}}
% make phone number
\newcommand{\phone}[2]{\bfseries \lr{\texttt{(+#1) #2}}}
% custom make title
\renewcommand{\maketitle}{
  \begin{center}
    {\Huge \bfseries \thetitle}\\
    \vspace{4em} \theauthor\\
    \vspace{.5em} \lr{Email: \mkemail{edu.siamak@gmail.com}}
  \end{center}
}
% `key: value` command
\newcommand{\kv}[2]{\textbf{#1:} {#2}}
\newcommand{\kvdash}[2]{\textbf{#1}  ---   \textit{#2}}
\newcommand{\codet}[1]{\lr{\texttt{#1}}}
%%
% tasks preferences
\settasks{label= }
%%
% clean page
\pagestyle{empty}


\begin{document}
\title{رزومه}
\author{سیامک آموزگار}
\maketitle

\section{مشخصات} %%%%%%%%%%%%%%%%%%%%%%%% section 1
\subsection{شخصی}
\begin{multicols}{2}
  \begin{tasks}
    \task\kv{نام}{\theauthor}
    \task\kv{تولد}{1378}
    \task\kv{تحصیلات}{لیسانس}
    \task\kv{محل سکونت}{کرج}
    \task\kv{وضعیت سربازی}{نامعلوم}
  \end{tasks}
  \columnbreak
  \begin{tasks}
    \task\kv{تماس}{\phone{98}{921-962-4341}}
    \task\kv{ایمیل}{\mkemail{edu.siamak@gmail.com}}
    \task\kv{لینکدین}{\href{https://www.linkedin.com/in/siamak-a-m-o/}
      {\texttt{linkedin.com/in/siamak-a-m-o}}}
    \task\kv{تلگرام}{\mktelegram{siamak4mo}}
  \end{tasks}
\end{multicols}
\subsection{پروژه ها}
\subsubsection{\lr{gitlab}}
\url{gitlab.com/SI.AMO} -- شامل تمام پروژه های جدیدم می شود
\subsubsection{\lr{github}}
\url{github.com/siamak4mo} -- شامل پروژهای قدیمی هم هست که دیگر آپدیت نمی شوند
\section{مهارت ها / تخصصی} %%%%%%%%%%%%%%%%%%%%%%%%%%%%%%% section 2
\subsection{زبان ها}
\subsubsection{\kvdash{C}{آشنایی متوسط}} پروژه های
\codet{my-small-c-projects} و \codet{my-small-arduino-projects}
\subsubsection{\kvdash{Golang}{آشنایی متوسط}}
یک پروژه چت مشابه \lr{IRC} به نام \codet{GoChat}
\subsubsection{\kvdash{Java}{ اولین زبان / آشنایی مختصر}}
با مفاهیم \lr{OOP} آشنایی دارم ولی از جاوا استفاده کمی داشتم
\subsubsection{\kvdash{Bash}{تسلط }}
تجربه تعدادی اسکریپت در پروژه: \codet{dot.scripts/scripts}
\subsubsection{\kvdash{Python}{آشنایی متوسط}}
استفاده برای حل مسائل مختلف مثل حل بازی های \lr{CTF} و مسائل دانشگاهی \\
مثل پروژه \codet{Dynamic Systems Plotting Tools} و غیره
\subsubsection{\kvdash{js - R - Matlab}{آشنایی مختصر}}
با زبان های \lr{R} و \lr{Matlab} در دانشگاه آشنا شدم و استفاده چندانی ندارم
\subsubsection{\textbf{زبان های Markup}}
\lr{HTML} - \lr{\LaTeX} - \lr{Markdown} - \lr{Rmarkdown}
\subsection{دیزاین پترن ها / فریم ورک ها}
\subsubsection{\lr{Design Patterns}}
آشنایی مقدماتی با دیزاین پترن ها
\subsubsection{\lr{Web}}
آشنایی با \lr{Rest API} و تجربه کمی از استفاده از
\lr{Django} (بطور خاص \lr{Django REST}) و \lr{Express.js}
\subsection{سیستم عامل / \lr{Virtualization}}
\subsubsection{\lr{LPIC}}
تسلط به \lr{LPIC-1}  -  آشنایی مقدماتی با \lr{LPIC-2}
\subsubsection{\lr{Docker}}
آشنایی با مفاهیم \lr{OS-level virtualization} - داکرایز کردن برنامه ها
- \lr{docker compose}
\subsection{دیگر}
\subsubsection{\lr{Git}}
آشنایی و استفاده از گیت - معمولا از \lr{magit} استفاده میکنم
\subsubsection{\lr{Database}}
تسلط به زبان \lr{SQL} و آشنایی با \lr{NoSQL} \lr{(MongoDB, GraphQL)}

آشنایی با مفاهیم \lr{relationship} در دیتابیس ها و استفاده از \lr{ORM}
\subsubsection{\lr{Automation}}
معمولا از \lr{Makefile} و اسکریپت های \lr{Bash} استفاده میکنم
\subsubsection{\lr{Devops}}
آشنایی مقدماتی با مباحث \lr{Devops} و \lr{GitLab CI}
\subsection{علاقه ها}
\subsubsection{\lr{Low-Level}}
با ابزارهایی مثل \lr{Ghidra, radare2, GDB} و استفاده از میکروکنترلر های \lr{AVR} آشنایی دارم
\subsubsection{نرم افزار آزاد / امنیت نرم افزار}
با مباحث \lr{CEH} آشنایی دارم
\section{تحصیلات} %%%%%%%%%%%%%%%%%%%%%%%%%%%%% section 3
\subsection{دانشگاه صنعتی شریف}
\textbf{B.Sc.} \textit{(2018 - 2022)} --- ریاضیات و کاربردها
[{\footnotesize \lr{MATHEMATICS AND APPLICATIONS}}] -
گرایش ریاضی نظری (محض)
\subsection{مرتبط}
\subsubsection{شبکه}
آشنایی با پروتکل های \lr{IP / TCP / UDP} - برخی پروتکل های لایه های پایین تر - پروتکل های وب و ایمیل
\subsubsection{ساختمان داده}
بیشتر مباحث کتاب \lr{CLRS} شامل ساختمان های داده - الگوریتم ها - آنالیز مقدماتی الگوریتم ها
\subsubsection{مدارمنطقی}
نمایش اعداد - جبر بول - \lr{latch}ها
\subsubsection{آمار / تحلیل رگرسیون}
تست های آماری - رگرسون خطی و غیر خطی - کار با زبان \lr{R}
\section{سوابق}
\latin{There's nothing here yet!}
\end{document}
